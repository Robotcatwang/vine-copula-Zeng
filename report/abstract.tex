\begin{cabstract}
  \renewcommand{\chapterlabel}{摘\hspace{2em}要}

 防范化解重大风险,特别是防控系统性金融风险被列为我国“三大攻坚战”之首。根据2019年中央经济工作会议相关
精神,防范金融市场异常波动和共振,打好防范化解系统性风险这一重点战役,是保持经济持续健康发展的重要保
障。近年来随着各金融部门的资产和信贷风险之间的协同流动不断增加,致使系统性金融风险呈现显著的跨部门传
导效应,这意味着如果只侧重于单个金融部门内部的风险传导,而忽视了机构间的关联性,就可能低估相互依赖
的金融机构对系统性金融风险的整体贡献,从而无法正确衡量金融系统中的风险溢出效应。同时“金融机构‘太大而
不能倒’的传统观念正逐步向‘太关联而不能倒’的思想转变”。正是由于证券市场中的各行业板块的股价日趋呈现出
交叉联动的特点,研究证券市场板块间风险传染路径、机制以及风险相关程度、影响力度,为政策制定及投资决定
提供参考建议,成为一个日渐紧迫的任务,越来越具有现实意义。

因此本文首先以CCA模型为基础对金融各板块的系统性风险进行计算。但是在计算中,本文并非直接使用板块指数
等整体性指标计算板块的系统性风险,而是通过化整为零,先计算各板块样本股的违约距离,再进行汇总得到板块
的系统性风险,尽可能减小系统性风险测度过程中存在的误差。然后对各部门系统性风险指标确定最优的R藤结构
形式,将金融板块的基本风险网络搭建出来,并进行相应的分析。最后确定并估计藤结构中节点与节点间边的
Copula类型及参数值,通过分析Copula的类型和相应的取值,分析板块间的尾部相依系数、无条件Kendall相关系
数和条件Kendall相关系数,分析得到板块间系统性风险的传染效应。

在研究过程中得出以下研究结论:(1)从各板块的系统性风险情况来看,在2015年前均处于较稳定的状态,但实
际上证券市场积累的杠杆、配资已经十分庞大,所以当证监会宣布彻查场外配资时,所有金融板块的系统性风险在
2015年下半年都有了显著的增加,其中尤以券商信托板块最为严重。不过随着国家政策的逐步推进,从2016年起各
板块的风险指标均有了显著的好转,但是各板块的风险指标波动浮动相较前五年依然有明显的增大。(2)从板块
间的风险传染结构来看,券商信托板块处于核心节点的位置,而银行板块处于整个结构的边缘位置,表明券商信
托板块属于金融板块中的风险核心,并且从Copula的类型来说,券商信托板块与其他板块间均为对尾部变化比较
敏感的Copula形式,说明当券商信托板块出现系统性风险损失时,其他板块都有可能受到系统性风险的感染。在
所有的条件板块中,券商信托板块出现的次数最多,保险板块其次,再次表明券商信托板块是金融板块系统性风
险传染过程中的关键点。 
  
  \bigbreak

  {\bfseries 关键词}:金融板块,系统性风险,CCA模型,藤Copula模型,风险传染
   

\end{cabstract}




\begin{eabstract}
Prevention and resolution of major risks, especially the prevention and control of systemic financial risks, has been listed as the first of the "three major battles" in China. According to the relevant spirit of the 2019 central economic work conference, to prevent abnormal fluctuations and resonance in the financial market, and to fight the key campaign of preventing and resolving systemic risks is an important guarantee for the sustained and healthy development of the economy. In recent years, with the increasing collaborative flows between assets and credit risks in various financial sectors, systemic financial risks have shown significant cross sector transmission effect, which means that if we only focus on risk transmission within a single financial sector and ignore the inter agency relationship, we may underestimate the overall contribution of interdependent financial institutions to systemic financial risks, So we can't measure the Risk Spillover Effect in financial system. At the same time, "the traditional concept of" too big to fail "of financial institutions is gradually changing to the idea of" too connected to fail ". It is precisely because the stock prices of various sectors in the securities market are increasingly showing the characteristics of cross linkage. It is an increasingly urgent task to study the risk contagion path, mechanism, risk correlation degree and impact strength among sectors in the securities market, so as to provide reference suggestions for policy making and investment decisions, which is becoming more and more realistic.



Therefore, based on CCA model, this paper firstly calculates the systematic risk of each financial sector. However, in the calculation, this paper does not directly use the overall indicators such as the plate index to calculate the systematic risk of the plate, but through breaking up into parts, first calculating the default distance of the sample stocks of each plate, and then summarizing the systematic risk of the plate, so as to reduce the error in the process of systematic risk measurement as much as possible. Then, the optimal r-vine structure is determined for the systematic risk indicators of each department, and the basic risk network of the financial sector is constructed, and the corresponding analysis is carried out. Finally, we determine and estimate the copula type and parameter value between nodes and the edge of nodes in rattan structure. By analyzing the copula type and corresponding value, we analyze the tail dependence coefficient, unconditional Kendall correlation coefficient and conditional Kendall correlation coefficient between plates, and analyze the infectious effect of the system risk between plates.



In the research process, the following conclusions are drawn: (1) from the perspective of the systematic risk of each sector, it was in a relatively stable state before 2015, but in fact, the leverage and capital allocation accumulated in the securities market have been very large, so when the CSRC announced a thorough investigation of OTC capital allocation, the systematic risk of all financial sectors increased significantly in the second half of 2015, especially in the financial sector The trust sector of securities companies is the most serious. However, with the gradual promotion of national policies, the risk indicators of each sector have improved significantly since 2016, but the fluctuation of risk indicators of each sector is still significantly increased compared with the previous five years. (2) from the perspective of the risk contagion structure between the plates, the trust plate of securities companies is at the core node position, while the bank plate is at the edge of the whole structure, indicating that the trust plate of securities companies belongs to the risk core of the financial plate, and from the type of copula, the trust plate of securities companies and other plates are all copula forms that are more sensitive to tail changes, indicating that when the credit of securities companies When there is systemic risk loss in Tuo sector, other sectors may be infected by systemic risk. In all the condition sectors, the securities trust sector is the most frequent, followed by the insurance sector, which once again shows that the securities trust sector is the key point in the process of systemic risk contagion in the financial sector.
  \\*

\end{eabstract}

\ekeywords{Financial Sectors,Systemic Risk,CCA Model,Vine-Copula Model,Risk Contagion }
