\begin{cabstract}
  \renewcommand{\chapterlabel}{摘\hspace{2em}要}

 2008年1月9日,自从中国黄金期货被首次推出之日以来,世界经济局势大幅波动,黄金期货主力连续合约
便出现数个跳空缺口,市场剧烈波动,很多投资者产生亏损并纷纷退出,市场流动性很差。随着商业银行
准入、夜盘交易等制度的颁布与实施,我国黄金期货市场参与者逐渐增多,市场流动性大幅度加强。但是
受到美元走势、外国突发事件等影响,市场波动仍非常剧烈,国际黄金期货市场价格大幅震荡波动。在此
背景下,要使黄金期货市场能够平稳有序地运行与发展,必须准确地度量黄金期货的风险,从而对风险进
行有效的管理,对未来中国黄金期货市场的发展大有裨益。

本文以上海期货交易所的黄金期货主力合约的结算价为基础来计算期货市场价格的日间波动测度,为了有
效地度量波动性,我们引入了具有动态尺度和形状参数的Generalized Autoregressive Score Model(GAS模型),并进一步基于不同边缘分布的条件异方差GARCH模型和EGARCH模型以及GAS的特例
Beta-t-EGARCH模型,探讨黄金期货市场价格波动风险,并将之与已有基于GAS模型的实证结果进行综合比较。
同时,采用传统的三种失败率检验法和GAS框架下独有的三种检验统计量对风险模型的度量效果进行回测检验。
最后对黄金期货的波动率进行5期预测。实证结果显示,GAS模型很好地拟合了我国黄金期货市场波动率序列,
并对后期的波动率具有较好的预测效果,同时GAS模型较GARCH模型对捕捉波动更敏感,能更好地拟合黄金期
货市场的波动性。

黄金期货波动性直接影响黄金期货交易成本和流动性风险,除了风险规避功能以外,黄金期货还是一种重
要的投资工具,可以对预期收益产生重要影响,投资者可以通过预测分析黄金的价格走向,选择合适的投
资方式来获取投资收益。期货市场的价格波动代表了投资者在市场信息变化时,对新信息作出的反应; 
研究黄金市场波动性有助于投资者把握市场行情变动,更加合理的指定交易策略。并且有助于监管部门更好
的监控市场状况,选择合适的调控手段,在适当的时机对市场进行调控,进而促进我国黄金期货市场监管体
系的完善,保障市场的健康平稳运行。我们的研究结果推动了新的GAS模型在我国黄金期货市场波动率测量、
金融衍生品定价或风险价值(VaR)和预期亏空(ES)度量的应用,为监管者和投资者提供一定的参考意见。 
  
  \bigbreak

  {\bfseries 关键词}:GAS,黄金期货,波动风险,滚动预测
   

\end{cabstract}




\begin{eabstract}
On January 9, 2008, since the first launch of gold futures in China, the world economic situation 
has fluctuated greatly, leading to several vacancies in the main consecutive contracts of gold 
futures. As a result, the market has fluctuated violently, and many investors have lost money and 
withdrawn, resulting in poor market liquidity. With the promulgation and implementation of such 
systems as commercial bank access and night trading, the number of participants in China's gold 
futures market has gradually increased and the market liquidity has been greatly enhanced. However,
affected by the trend of the us dollar and foreign emergencies, market volatility is still very 
severe, and the international gold futures market prices fluctuate greatly. In this context, in 
order to ensure the smooth and orderly operation and development of the gold futures market, it is 
necessary to accurately measure the risk of gold futures, so as to effectively manage the risk, 
which is of great benefit to the future development of China's gold futures market.

Based on the Shanghai futures exchange's main gold futures contract settlement price calculated on 
the basis of the futures market price volatility measure in the day, and in order to effectively 
measure volatility, we introduced the dynamic scale and shape parameters of Generalized 
Autoregressive Score Model (GAS) Model, and further the conditional heteroscedasticity GARCH Model 
based on different marginal distribution and EGARCH Model as well as a special case of the GAS 
Beta-t-EGARCH Model, to explore gold futures market price volatility risk, It is compared with the 
empirical results based on GAS model. At the same time, the traditional three failure rate test 
methods and the unique three test statistics under the GAS framework were used to carry out 
backtesting on the measurement effect of the risk model. Finally, the volatility of gold futures is
predicted in 5 stages. The empirical results show that the GAS model fits the volatility sequence 
of China's gold futures market well, and has a good prediction effect on the later volatility. 
Meanwhile, the GAS model is more sensitive to capturing volatility than the GARCH model, and can 
better fit the volatility of the gold futures market.

Gold futures volatility directly affect the cost of gold futures trading and liquidity risk, in 
addition to the risk aversion function, gold futures is a kind of important investment tools, can 
be a important impact on earnings, investors can be predicted through the analysis of the price of 
gold, to choose the right means of investment to obtain investment returns. The price fluctuation 
of the futures market represents the reaction of investors to the new information when the market 
information changes. Studying the volatility of gold market is helpful for investors to grasp the 
changes of market conditions and specify more reasonable trading strategies. What's more, it is 
helpful for regulators to better monitor market conditions, select appropriate regulatory means and regulate the market at appropriate times, so as to improve the regulatory system of China's gold 
futures market and ensure the healthy and stable operation of the market. Our research results to 
promote the new GAS model in China's gold futures market volatility measurement, financial 
derivatives pricing or value at risk (VaR) and expected shortfall (ES), the application of 
measurement, provides some reference opinions for regulators and investors.
  \\*

\end{eabstract}

\ekeywords{GAS,gold futures,volatility risk,rolling forecast }
